\documentclass[a4paper,11pt]{article}
\usepackage[utf8]{inputenc}
\usepackage[swedish]{babel}
\usepackage{graphicx}
\usepackage[colorinlistoftodos]{todonotes}
\usepackage[affil-it]{authblk}
\usepackage{verbatim}
\usepackage{microtype}
\usepackage{sidecap}
\usepackage{amssymb}
\usepackage{wrapfig}
\usepackage{tocloft}
\renewcommand{\cftsecleader}{\cftdotfill{\cftdotsep}}

\usepackage{animate}
\usepackage{amsthm}
\usepackage{hyperref}
\usepackage{gensymb}
\usepackage[hypcap=false]{caption}
\usepackage[version=4]{mhchem}
\usepackage[swedish]{babel}
\emergencystretch=1em
\usepackage[
backend=biber,
style=numeric,
sorting=none
]{biblatex}
\usepackage{color}
\usepackage{hyperref}
\hypersetup{colorlinks,citecolor=red,linkcolor=red}

\begin{document}

\begin{titlepage}
	\centering
	\includegraphics[width=0.6\textwidth]{Bilder/logo.png}\par\vspace{1cm}
	\vspace{1.5cm}
	{\huge\bfseries Internetshistoria\par}
	\vspace{2cm}
	{\Large\itshape Adam Blomberg\par}
	\vfill
\animategraphics[autoplay,loop,width=1\textwidth]{12}{Bilder/grandmaSurf/grandma-}{0}{18}
	\vfill

% Bottom of the page
	{\large \today\par}
\end{titlepage}

\tableofcontents
\newpage

\section{Introduktion}
För närvarande är internet bland det hetaste ämnet inom nutida data-
och populärvetenskap. Webbens enorma framgång de senaste åren är väldigt svår
att undgå. Det finns en uppsjö av webbartiklar inom massmedia, och termer
inkluderar kryptiska sammansatta bokstäver och tecken som ofta börjar med
``http://'' blir allt vanligare. Vad som fattas från de nuvarande beskrivningen
av nätet är en diskussion om intenets 30-åriga historia av forskning och
utveckling som har skapat de underliggande teknologier av vilka webben är
baserad på. Mycket av denna grund var lagd på 1960-talet av Douglas Engelbart.
Jag kommer att göra en grundlig sammanfattning av internets historia från tidiga
60-tal till nutid för att få en inblick i utvecklingen av det mytomspunna
artificiella nätverk vi kallar Internet.

\section{Där det hela började}

Det var tidigt 60-tal och Joseph Licklider hade redan storslagna planer om ett
intergalaktiskt nätverk. Hans vision var ett globalt sammanlänkat nätverk av
datorer som alla med tillgång till nätverket snabbt skulle kunna hämta data och
program från alla webbsidor. Man skulle kunna säga att han var mycket före sin
tid eftersom att hans vision sedan blev verklighet under 2000-talet. Syftet var
att skapa ett säkert och decentraliserat nätverk som inte kunde slås ut av något
främmande främmande land, främst Sovjetunionen. Arpanet skapades av den
amerikanska forskningsanstalten \textit{Advanced Research Projects Agency,} ARPA
år 1969. Det fanns endast en omfattande aktör inom telekombranchen vilket var
AT\&T som nästan hade monopol, de ägde telefonledningarna som så småningom om
skulle komma att användas till att koppla ihop datorerna i ARPANET.
Universitetet UCLA i Kaliforninen skickade det första meddelandet vilket var
kommandot "LOG", som skulle användas för att fjärrlogga in på en annan dator.
Mottagardatron krashade dock innan den hunnit ta emot hela kommandot,
meddelandet som skickades blev istället "LO". Det är en historisk händelse då
datorer för första gången har pratat med varandra.

På tal om Sovjet, de hade en liknande ide. Kort därefter att J. Licklider talat
om sina tankar, börjar Viktor Glushkov berätta om sin liknande plan med namnet
OGAS. Detta går att jämnföra med Rymdkapplöpningen som ägde rum mellan 50- till
60-tal. Det var tänkt att OGAS skulle var uppdelat i tre lager, enligt Glushkov.
Nätverket skulle ha sitt styre i Moskva med totalt 20.000 sammankopplade datorer
genom Sovjetunionen. Datan skulle passera genom det redan existerande
telefonledningarna på en frekvens som då inte var utnyttjad av telefonsamtalen,
men ju mer data som behöver passera genom telefonledningarna desto större
behöver frekvensspannet att vara. Till en viss punkt kan du inte göra spannet
större och du har då kommit till en slutpunkt. Under denna tid var inte detta
ett stort problem, men med tiden började vi ladda ned mer och strömma video. Det
är därför fiber används istället nu för tiden eftersom att den har mycket större

\begin{wrapfigure}{r}{0.7\textwidth}
  \vspace{-7pt}
  \centering
  \includegraphics[width=1.0\linewidth]{Bilder/bandbredd.jpg}
  \captionof{figure}{Som vi kan se i denna illustration används majoriteten av
    bandbredden i telefonledningarna till data överföring vilket omfattar
    faxmedelanden, men främst internetanvänding.}
\end{wrapfigure}

Tanken var att de skulle kunna kommunicera med varandra på en decentraliserad
nivå -- utan att behöva passera genom Moskva. Det var senare tänkt att under
etapp två -- skulle resterande världen anslutats, främst Europa och Asien. USA
var inte ens på kartan på grund av de dåliga relationerna mellan Sovietunionen
och USA. Alla fabriker och kontorskomplex skulle vara uppkopplat. Det var en
genialisk idé enligt Glushkov själv. Genom nätverket skulle framförallt
betalningar ske digitalt, ungefär som det sker idag via internetbankerna.
Projektet blev dock aldrig en verklighet eftersom att Glushkov inte lyckades
finansiera sitt projekt, han fick inte vidare finansiering av regeringen. Vid
det laget har Arpanet redan slagit rot, som senare kommer att bli det vi idag
kallar internet. Det var först 1969 som det första meddelandet skickades på
Arapanet.
\end{document}
